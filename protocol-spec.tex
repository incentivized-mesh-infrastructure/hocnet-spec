 \documentclass[11pt]{article}
\usepackage[linktocpage=true]{hyperref}
\usepackage[utf8]{inputenc}
\usepackage[T1]{fontenc}
\usepackage{fixltx2e}

\title{\textbf{Hocnet a payment based mesh network protocol built on Batman-Adv}}
\date{2016-12-10}
\begin{document}

\maketitle

\section{Work in progress}

This paper is incomplete, feel free to provide feedback via Github issues

\tableofcontents

\section{Abstract}

As the number of connected individuals and devices expands the `last mile` continues to be the greatest challenge both in the connected and developing worlds, representing a disproportional portion of the cost and difficulty of connecting the world.

Ad-Hoc networks are a type of network where there is no dedicated infrastructure such as switches or routers and instead these tasks are handled by the participating users in a distributed fashion. Mesh networks are a derivative of Ad-Hoc networks in that dedicated hardware for routing is used and encouraged but not ultimately required for the functionality of the network. 

In this paper we propose a plethora of additions to the existing mesh network protocol Batman-Adv with the purpose of creating a 'decentralized last mile' where each node competes to be paid to carry traffic via the most efficient route it can use. Furthermore we will encourage the use of so called `Ad-Hoc infrastructure` where users may implement dedicated hardware such as directional antennas, IR links, or simply mundane wires to produce better and more profitable routes for their nodes to advertise.

We hope that by lowering the barrier of entry to providing connectivity services we can both decrease the cost of connecting the developing world and provide an environment that fosters competition in otherwise monopolistic last mile infrastructure in developed countries.


\section{Batman-Adv}

    Batman-Adv is the layer 2 version of the original BATMAN protocol outlined in \cite{batman}. Conceptually the protocol is very simple. Every originator period every node participating in the network broadcasts an originator message containing a MAC address, which identifies the node that `originated` the message as well as the MAC address of the last node to forward it.  When an originator message is broadcast every node that receives the broadcast updates the transmission quality (TQ) field with it's own estimation of the link quality between itself and the node that sent the originator message. Note that the TQ field is not simply replaced by every hop, but is updated using several factors that boil down to a single field in the originator message for which 'higher is better' for more details see appendix A in \cite{batroam}.
    
    So as the originator messages spread through the network each node checks each originator message it receives against it's own local routing table, either to be added or updated if the TQ advertised by the recived originator message is better or the sequence number of the message is higher. Since originator messages contain the last hop to forward the message this data combined with TQ and sequence number is used to maintain an up to date routing table with the best first hop to any other node in the network.
    
    To send data to any other node it is sent to the best first hop, who forward that data in turn to it's own best first hop to the destination and so on until the data reaches it's destination node.
    
    Batman-Adv is implemented as an up-streamed Linux kernel module with several optimizations from the original concept, including network coding to batch the broadcast of originator messages and several optimizations focusing on allowing hybrid networks containing traditional Ethernet links as well as wireless links \cite{catwoman}. Perhaps the most promising feature of Batman-Adv is that is remains competitively performant when compared to other mesh protocols \cite{meshperf}


\section{Modifications to Batman-Adv protocol spec}

	\subsection{Addition of cost criteria}

		Specifically we propose a new field be added to BATMAN originator messages, this field is simply a half-precision floating point value for the cost of transmitting a pre-arranged number of packets in a single direction, an originator creating a new originator message for broadcast into the network would initialize this value to their 'hop cost'. A node rebroadcasting a received originator message would update this field by adding their own hop cost to the existing value before forwarding.

	        The process for selecting which originator message to retain and rebroadcast is likewise updated to account for the cost field. Instead of choosing the best originator message using the TQ field as the sole criteria the ratio of cost to TQ is chosen.

                Protocol overhead will be unbilled, this provides an unfortunate avenue for the use of stenography to encode free data. We propose a quick solution by limiting the frequency of overhead packet forwarding to the default issuing rate. In this way a malicious client could in theory reduce their broadcast rate to encode data for free, but in exchange said client would actually be reducing the overhead on the network rather than adding to it.

                Furthermore, for the time being all bandwidth in either direction is paid for by the node initiating the connection, this is useful in that it simplifies the billing structure for the intended use of last mile consumer connectivity but is not a requirement of the protocol design.

	\subsection{Payment chains} \label{paychains}

               The design of BATMAN, as well as any reasonably low overhead mesh protocol, specifically avoids requiring any node to have full knowledge of a route it is using, from the perspective of any given member of the mesh only the next hop and the cost required to reach the destination are known. To negotiate with each node along any given route to figure out who to compensate for routing packets is not only unsupported by BATMAN but is impossible to do efficiently. Mesh nodes would spend a significant amount of time and funds paying intermediate nodes to forward negotiation packets to every link in the route until a connection was finally open. To make a bad situation worse, real world testing shows that in worse case scenarios routes can flip as frequently as once every several seconds \cite{meshflip}. While protocol modifications can improve this alarming figure it's not realistic to expect good performance out of a protocol that requires negotiating with many nodes individually. Furthermore paying each node in a route individually raises the issue of very infrequent paths, as routes move due to network conditions it's entirely possible many nodes will owe funds to each other without the individual amounts ever exceeding the fees for issuing the transactions.

               The solution we propose involves having route prices advertised as 'lump sums', where the cost of the entire route is paid to the first node in that route and it's expected that each node will pay forward all except it's own share. Since nodes are much more likely to send significant amounts of traffic over each adjacent node as opposed to any given node in a network this should result in the vast majority of traffic being feasible to pay out even with a significant amount of route flipping. We will spend the next several sections exploring what guarantees and security we can provide before addressing payment in further detail in \ref{pay}

    \subsection{Weak black-hole attack protection} \label{blackhole}
    
               As noted in the previous section we refer to nodes advertising false routes, this is just one subset of what is known as a block-hole attack. A malicious node can falsely advertise routes that it is unable to complete, grinding the network to a halt as traffic is directed into a 'black-hole' from which is does not escape \cite{routesec}. While we can easily prevent the aforementioned attack as well as many other scams by having all bandwidth purchased on credit we can not use this method to prevent a well connected node from attracting more than its share of traffic by fudging a better TQ on each originator message it passes.
               
               Since Hocnet is expected to use and encourage the use of Ethernet bridges, IR bridges, and other ad-hoc infrastructure it's possible for a node to have a near perfect TQ to another node at an arbitrary distance. Making it impossible for any given node to determine the validity of an originator message without direct communication with the originator.
               
               To resolve this problem using constant originator message space we propose the addition of time stamps in originator messages. The originator will place a signed time stamp into originator messages. After some predetermined period t a node forwarding an originator message will be required to sign the current TQ and route cost with its public key and attach this data to the originator message. This process will be repeated for each time period up to a small but arbitrary number of times m.
               
               originator messages found to be lacking these values after the prescribed time will be dropped. This allows any node receiving an originator message to verify that the route cost and TQ have decreased and increased respectively since the last checkpoint. The strength of the security this provides against malicious originator messages is inversely proportional to t, where a small enough time period will devolve into the same O(n) scheme discarded earlier in this section. 
               
               While this algorithm provides no strong guarantees we hope to tune it such that black hole attacks aren't effective on anything but very small subjects of the network. 
    
    \subsection{Clock synchronization bootstrapping}
    
               The above section requires a that all nodes maintain a synchronized clock. TODO describe method to bootstrap clock sync passively using observed neighbor originator message time stamps. TODO replace originator message number with time stamp?
    
    
    \subsection{Key exchange bootstrapping}
    
               To prevent impersonation it's critical that two nodes sharing a MAC address but using a different pubkey can not exist. We propose a simple solution where any originator message using an identical MAC address to an existing originator message but a different key will simply be dropped by the receiving node. Combined with a reasonable policy for evicting old entries for nodes not seen on the order of hours should be sufficient. 
               
               TODO think more about possible attacks here
    
    \subsection{Potential optimizations using hash-chains}
    
               Asymmetric encryption is a costly operation, even though we only propose security features on network overhead packets and leave security of the actual traffic to the layer 3 protocol our current proposal involves message verification operations proportional to the network size every originator message period, even using RSA with small keysizes a node modest node will struggle as network size increases.
               
               Using one way hash-chains we can provide a similar level of security with a much less costly cryptographic primitive. In a one way hash chain a node chooses a random seed S and uses a secure hash function H to create a hash chain of length n. 
               
               \[H(S) = V_n\] 
               
               Where V\textsubscript{n} is the end of the chain. From that point V\textsubscript{n-1} ... V\textsubscript{0} are computed by.
               
               \[H(V_{n+1}) = V_n\] 
               
               Using this method a node generates an arbitrary but finite chain of length n. Data can be symmetrically signed or otherwise encrypted with V\textsubscript{i}, which is then transmitted to the network. Some period of time later V\textsubscript{i-1} is revealed to the network and can be used to verify the authenticity of the earlier message. By maintaining a longer history a node can verify that a given series of messages must originate from the same sender.
               
               By embedding a chain V\textsubscript{n} and V\textsubscript{i+1} in each originator message V\textsubscript{n} can be used to to verify the previous originator message sent by that node, secured with V\textsubscript{i-1} and V\textsubscript{i+1} can be used to verify the originator message containing it once V\textsubscript{i+2} is released by the next originator message. When V\textsubscript{n} is finally reached a new chain is generated and it's V\textsubscript{0} is placed in an originator message that is verified using S.      
               
               
               In the name of additional efficiency we propose the use of the same secure hash function H for generating the message signatures. Where M\textsubscript{sig} represents the originator message signature and || represents binary append.
               
               \[M_{sig} = H(Timesamp||MAC||V_n||V_{n+1})\]
               
               Once V\textsubscript{n+1} has been revealed in the next originator message this signature can be computed and verified. By requiring the next originator message to verify the current one we effectively add a latency of one originator message period to all overhead operations if we are to verify the contents of each message before using them. It is possible that the contents of the message could be used initially and then discarded if the message fails verification, at the very least it can be confirmed that.
                     
                     \[H(V_{n-1}) == V_n\]
                     
               Which proves that some valid version of the originator message from that node exists in the network, although it's possible that the received version has been modified in transit until verification is complete. To reduce this latency we propose that if nodes only see one version of a given originator message with a valid V\textsubscript{n} it should be accepted immediately but placed in a queue for verification \cite{spins,hash}.
               
               \subsubsection{Persistence of Trust}
               
                      While hash chains offer significantly better performance they do not provide an easy way to verify that you are speaking to the same node after instances of power loss or even intermittent connection loss. Resuming verification is possible after receiving more than one correct originator message but without a constant pubkey there is no way to verify that the identity of those node has not changed while the node was away. Since trust is an integral part of the billing system this means that we can not eliminate the need for asymmetric key cryptography entirely. 
                      
                      On the other hand, unless you have a billing relationship with a node trust is not affected, meaning pubkeys only need to be shared with adjacent nodes. Even better only one asymmetric key operation needs to be performed to verify that the owner of the pubkey is the owner of a given hash chain, after which all billing messages can be verified using only the current V\textsubscript{n} for any given node.
               
               \subsubsection{The large network problem}
                        
                        In a case where the longest path through the network is longer than the originator message period it's possible to perform a rather complex attack. An attacker would have a low latency connection across the diameter of the network, they would pick up originator messages at one end of the network, wait for the next message to reveal the key, then use that key to forge a false originator message and broadcast it before the original message reaches that point in the mesh. Victim nodes would be faced with two valid, conflicting originator messages.
                        
                        A proposed solution is to use the checkpoints previously proposed in \ref{blackhole}, when a node producing a checkpoint in an originator message produces their signature M\textsubscript{chk\textsubscript{n}} which is the n\textsuperscript{th} checkpoint signature.
                        
                        \[M_{chk_n} = H(V_{chk_n}||V_{chk_{n+1}}||Timestamp_{chk_n}...Price_{chk_n}||TQ_{chk_n})\]
                        
                        By signing the original contents of the message as well as the checkpoint contents no additional space is used because the hash output is a fixed size, but it becomes possible to verify what version of each originator message each saw exactly, as opposed to only the price and TQ observed. Provided the checkpoint interval and maximum number of checkpoints remains properly tuned for the maximum network diameter controlled by the originator message TTL a checkpoint should exist on any far flung originator message either before the next key reveal or too shortly after to reliably attack.
                                                
               \subsubsection{Dealing with attackers}
               
                      Using asymmetric key cryptography it was feasible to simply drop originator messages that failed to validate, while it's still possible to drop obviously invalid originator messages where the revealed V\textsubscript{n} does not match H(V\textsubscript{n-1}) various race condition attacks are possible. 
                          
                      A new verification queue, that only updates the routing table after verification for nodes with originator messages that failed to verify is possible, but opens up the possibility of using the solution as an attack in and of itself to delay updates about a specific node in the network. Since we assume ad-hoc infrastructure will make up the backbone of the network increased originator message periods are unlikely to re-route traffic, which would be the major motivation for such an attack. For the time being we leave this problem open.
                
\section{Payment models} \label{pay}

    Payment negotiation at the mesh level was mentioned in \ref{paychains} without much detail beyond the method. This is an intentional chose to abstract as many of the details of billing as possible into a user-space program, in this way we can embed a payment negotiation protocol into Hocnet at a higher level instead of being forced to write it into the kernel itself. The billing program, preliminarily named Penguin (backronym to be determined), requires read only access to the routing table, the ability to send messages to adjacent nodes, and the ability to block routing to a given adjacent node.
    
    During the initial key exchange a number of billing codes can be encoded in the message, advertising what methods of payment any given node can accept. It's then up to the node initiating traffic to either find a mutually acceptable payment method of blacklist the node. Billing info messages will have a hash signature as well as a symmetric one, allowing for easy verification although not easy generation. In the worst case a well connected node with many neighbours that has not yet built any trust may have to perform a large number of asymmetric key operations until trust has been established and the period of billing messages can increase.

    \subsection{Cryptocurrency}
    
          TODO read about lightning network
    
    \subsection{Semi centralized cryptocurrency}
    \subsection{Semi centralized traditional currency}

\medskip


\begin{thebibliography}{1}



\bibitem{batman}
David Johnson, Ntsibane Ntlatlapa, and Corinna Aichele.
\textit{A simple pragmatic approach to mesh routing using BATMAN} (2008)

\bibitem{catwoman}
Cigno, R., and Daniele Furlan.
\textit{Improving BATMAN Routing Stability and Performance} (2011)

\bibitem{batroam}
Quartulli, Antonio, and Renato Lo Cigno.
\textit{Client announcement and Fast roaming in a Layer-2 mesh network} (2011)

\bibitem{meshperf}
Murray, David, Michael Dixon, and Terry Koziniec.
\textit{An experimental comparison of routing protocols in multi hop ad hoc networks} Telecommunication Networks and Applications Conference (ATNAC), 2010 Australasian. IEEE, 2010.

\bibitem{meshflip}
Britton, Matthew, and Andrew Coyle.
\textit{Performance analysis of the BATMAN wireless ad-hoc network routing protocol with mobility and directional antennas} Military Communications and Information Systems Conference (MilCIS), 2011. IEEE, 2011.

\bibitem{sead}
Hu, Yih-Chun, David B. Johnson, and Adrian Perrig. 
\textit{SEAD: Secure efficient distance vector routing for mobile wireless ad hoc networks} Ad hoc networks 1.1 (2003): 175-192.

\bibitem{spins}
Perrig, Adrian, et al. 
\textit{SPINS: Security protocols for sensor networks} Wireless networks 8.5 (2002): 521-534.

\bibitem{hash}
Hu, Yih-Chun, Markus Jakobsson, and Adrian Perrig. 
\textit{Efficient constructions for one-way hash chains} International Conference on Applied Cryptography and Network Security. Springer Berlin Heidelberg, 2005.

\bibitem {routesec}
Lundberg, Janne. 
\textit{Routing security in ad hoc networks} Helsinki University of Technology, http://citeseer. nj. nec. com/400961. html (2000).

\end{thebibliography}

\end{document}
